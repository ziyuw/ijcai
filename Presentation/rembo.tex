\documentclass[grey]{beamer}
% Class options include: notes, notesonly, handout, trans,
%                        hidesubsections, shadesubsections,
%                        inrow, blue, red, grey, brown
%dvips -Ppdf -tletter -G0 -o paper.ps paper.dvi

\usepackage{amsmath,amsthm,amssymb}
%\usepackage{amsfonts}

\usepackage{color}
\definecolor{Blue}{rgb}{0.9,0.3,0.3}

%http://www-db.stanford.edu/~manku/latex.html
%The itemize environment can be replaced by:
\newcommand{\squishlist}{
   \begin{list}{$\bullet$}
    { \setlength{\itemsep}{0pt}      \setlength{\parsep}{3pt}
      \setlength{\topsep}{3pt}       \setlength{\partopsep}{0pt}
      \setlength{\leftmargin}{1.5em} \setlength{\labelwidth}{1em}
      \setlength{\labelsep}{0.5em} } }

\newcommand{\squishlisttwo}{
   \begin{list}{$\bullet$}
    { \setlength{\itemsep}{0pt}    \setlength{\parsep}{0pt}
      \setlength{\topsep}{0pt}     \setlength{\partopsep}{0pt}
      \setlength{\leftmargin}{2em} \setlength{\labelwidth}{1.5em}
      \setlength{\labelsep}{0.5em} } }

\newcommand{\squishend}{
    \end{list}  }

%Example usage: \squishlist    %% \begin{itemize}
%\item First item
%\item Second item
%\squishend     %% \end{itemize}

\newcommand{\denselist}{\itemsep 0pt\topsep-6pt\partopsep-6pt}

%% a trick that makes the title take up less space for many style files (but not article)
%\addtolength{\titlebox}{-1.8cm}

%% densify spacing in bibliographies
\newcommand{\bibfix}{%    PUT \bibfix in file.bbl after first line
    \setlength{\parsep}{\parskip}%
    \setlength{\itemsep}{0cm}%
    \setlength{\topsep}{\parskip}%
    \setlength{\parskip}{0cm}%
    \setlength{\partopsep}{0cm}%
    \setlength{\listparindent}{\parindent}%
    \setlength{\labelwidth}{10pt}%
    \setlength{\labelsep}{0pt}%
    \setlength{\leftskip}{0pt}%
    \setlength{\leftmargin}{0pt}%
}

%% change margins
%\setlength{\textwidth}{7in}
%\setlength{\textheight}{8.75in}
%\setlength{\oddsidemargin}{-0.25in}
%\setlength{\evensidemargin}{-0.25in}
%\setlength{\headsep}{10pt}

%Use changebar.sty  to track changes.

%Saving space: see
%   http://www-h.eng.cam.ac.uk/help/tpl/textprocessing/squeeze.html

%Page layout info:
%   http://amath.colorado\edu/documentation/LaTeX/reference/layout.html


%%%%%%%%%% Code listings



%Latex
%\documentstyle[fleqn,psfig,epsfig]{article}
%\documentstyle[psfig]{article}
%\setlength{\textwidth}{6.5in}
%\setlength{\oddsidemargin}{0in}
%\setlength{\textheight}{8.5in}
%\setlength{\headheight}{0in}
%\setlength{\headsep}{0in}
%\setlength{\parindent}{0in} % block style
%\setlength{\parskip}{0.3cm}

% % \newtheorem{example}{Example}[section]
% % \newtheorem{thm}{Theorem}[section]
% % \newtheorem{cor}{Corollary}[section]
% % \newtheorem{defn}{Definition}[section]
% % \newenvironment{mythm}{{\bf Theorem}}{}
% % \newenvironment{myproof}{{\bf Proof}}{}

%http://www.maths.tcd.ie/~dwilkins/LaTeXPrimer/Theorems.html
%\newenvironment{proof}[1][Proof]{\begin{trivlist}
%\item[\hskip \labelsep {\bfseries #1}]}{\end{trivlist}}

% make qed symbol a solid square
%\renewcommand{\qed}{\mbox{$\hrulefill \blacksquare $}}

%http://everything2.com/title/tombstone
%\renewcommand{\qed}{\hfill \nobreak \ifvmode \relax \else
%    \ifdim\lastskip<1.5em \hskip-\lastskip
%    \hskip1.5em plus0em minus0.5em \fi \nobreak
%    \vrule height0.4em width0.4em depth0.25em\fi}


%\newcommand{\subsubsubsection}[1]{\paragraph{#1}}
\newcommand{\choice}[2]{\left(\!\!\! \begin{array}{c} #1 \\ #2\end{array} \!\!\!\right)}
%\newcommand{\half}{\frac{1}{2}}
\newcommand{\half}{\frac{1}{2}}
%\newcommand{\defeq}{\stackrel{\rm def}{=}}
\newcommand{\defeq}{:=}
%\newcommand{\real}{{\rm I\hspace{-0.2em}R}}
\newcommand{\real}{\mathbb{R}}
%\newcommand{\indep}{\perp}

\newcommand{\given}{\|}
\newcommand{\indep}[2]{{#1} \perp {#2}}
\newcommand{\condindep}[3]{{#1} \perp {#2} | {#3}}
\newcommand{\condindepG}[3]{{#1} \perp_G {#2} | {#3}}
\newcommand{\condindepP}[3]{{#1} \perp_p {#2} | {#3}}
\newcommand{\depend}[2]{{#1} \not \perp {#2}}
\newcommand{\conddepend}[3]{{#1} \not \perp {#2} | {#3}}

%\newcommand{\trans}[1]{{#1}^{\mathtt{T}}}
\newcommand{\trans}[1]{{#1}^{T}}
\newcommand{\inv}[1]{{#1}^{-1}}

\newcommand{\ra}{\rightarrow}
\newcommand{\lra}{\leftrightarrow}
\newcommand{\Ra}{\Rightarrow}
%\newcommand{\rv}{r.v.}
\newcommand{\la}{\leftarrow}
\newcommand{\tr}{\mathrm{tr}}
\newcommand{\st}{\; \mathrm{s.t.} \;}
%\newcommand{\det}{\mathrm{det}}
\newcommand{\size}{\mathrm{size}}
\newcommand{\trace}{\mathrm{trace}}

%\newcommand{\do}{\mathrm{do}}
\newcommand{\pemp}{p_\mathrm{emp}}
\newcommand{\dom}{\mathrm{dom}}
\newcommand{\bel}{\mathrm{bel}}
\newcommand{\dsep}{\mathrm{dsep}}
\newcommand{\sep}{\mathrm{sep}}
\newcommand{\entails}{\models}
\newcommand{\range}{\mathrm{range}}
\newcommand{\myspan}{\mathrm{span}}
\newcommand{\nullspace}{\mathrm{nullspace}}
\newcommand{\adj}{\mathrm{adj}}
\newcommand{\pval}{\mathrm{pvalue}}
\newcommand{\NLL}{\mathrm{NLL}}


\newcommand{\betadist}{\mathrm{Beta}}
\newcommand{\Betadist}{\mathrm{Beta}}
\newcommand{\bernoulli}{\mathrm{Ber}}
\newcommand{\Ber}{\mathrm{Ber}}
\newcommand{\Binom}{\mathrm{Bin}}
\newcommand{\NegBinom}{\mathrm{NegBinom}}
\newcommand{\binomdist}{\mathrm{Bin}}
\newcommand{\cauchy}{\mathrm{Cauchy}}
\newcommand{\DE}{\mathrm{DE}}
\newcommand{\DP}{\mathrm{DP}}
\newcommand{\Dir}{\mathrm{Dir}}
%\newcommand{\discrete}{\mathrm{Discrete}}
\newcommand{\discrete}{\mathrm{Cat}}
\newcommand{\Discrete}{\discrete}
\newcommand{\expdist}{\mathrm{Exp}}
\newcommand{\expon}{\mathrm{Expon}}
\newcommand{\gammadist}{\mathrm{Ga}}
\newcommand{\Ga}{\mathrm{Ga}}
\newcommand{\GP}{\mathrm{GP}}
\newcommand{\GEM}{\mathrm{GEM}}
\newcommand{\gauss}{{\cal N}}
\newcommand{\erlang}{\mathrm{Erlang}}
%\newcommand{\IG}{\mathrm{InvGam}}
\newcommand{\IG}{\mathrm{IG}}
\newcommand{\IGauss}{\mathrm{InvGauss}}
\newcommand{\IW}{\mathrm{IW}}
\newcommand{\Laplace}{\mathrm{Lap}}
\newcommand{\logisticdist}{\mathrm{Logistic}}
\newcommand{\Mu}{\mathrm{Mu}}
\newcommand{\Multi}{\mathrm{Mu}}
%\newcommand{\Multin}{\mathrm{Mun}}
%\newcommand{\Mun}{\mathrm{Mun}}
\newcommand{\NIX}{NI\chi^2}
\newcommand{\GIX}{NI\chi^2}
\newcommand{\NIG}{\mathrm{NIG}}
\newcommand{\GIG}{\mathrm{NIG}}
\newcommand{\NIW}{\mathrm{NIW}}
\newcommand{\GIW}{\mathrm{NIW}}
%\newcommand{\MVNIW}{\mathrm{MVNIW}}
\newcommand{\MVNIW}{\mathrm{NIW}}
\newcommand{\NW}{\mathrm{NWI}}
\newcommand{\NWI}{\mathrm{NWI}}
%\newcommand{\MVNIG}{\mathrm{MVNIG}}
\newcommand{\MVNIG}{\mathrm{NIG}}
\newcommand{\NGdist}{\mathrm{NG}}
\newcommand{\prob}{p}
\newcommand{\Poi}{\mathrm{Poi}}
\newcommand{\Student}{{\cal T}}
\newcommand{\student}{{\cal T}}
\newcommand{\Wishart}{\mathrm{Wi}}
\newcommand{\Wi}{\mathrm{Wi}}
\newcommand{\unif}{\mathrm{U}}
\newcommand{\etr}{\mathrm{etr}}


%\newcommand{\dim}{\mathrm{dim}}
\newcommand{\mse}{\mathrm{mse}}
\newcommand{\pon}{\rho}
\newcommand{\lse}{\mathrm{lse}}
\newcommand{\softmax}{\calS}
\newcommand{\soft}{\mathrm{soft}}
\newcommand{\cond}{\mathrm{cond}}
\newcommand{\sign}{\mathrm{sign}}
\newcommand{\sgn}{\mathrm{sgn}}
\newcommand{\iid}{\mbox{iid}}
\newcommand{\mle}{\mbox{mle}}
\newcommand{\myiff}{\mbox{iff}}
\newcommand{\pd}{\mbox{pd}}
\newcommand{\pdf}{\mbox{pdf }}
\newcommand{\cdf}{\mbox{cdf}}
\newcommand{\pmf}{\mbox{pmf}}
\newcommand{\wrt}{\mbox{wrt}}
\newcommand{\matlab}{{\sc MATLAB}}
\newcommand{\NETLAB}{{\sc NETLAB}}
\newcommand{\MLABA}{\mbox{PMTK}}
\newcommand{\BLT}{\mbox{PMTK}}
\newcommand{\PMTK}{\mbox{PMTK}}
\newcommand{\mywp}{\mathrm{wp}}

\newcommand{\KLpq}[2]{\mathbb{KL}\left({#1}||{#2}\right)}
\newcommand{\KL}{\mathbb{KL}}
\newcommand{\MI}{\mathbb{I}}
\newcommand{\MIxy}[2]{\mathbb{I}\left({#1};{#2}\right)}
\newcommand{\MIxyz}[3]{\mathbb{I}\left({#1};{#2}|{#3}\right)}
\newcommand{\entrop}{\mathbb{H}}
\newcommand{\entropy}[1]{\mathbb{H}\left({#1}\right)}
\newcommand{\entropypq}[2]{\mathbb{H}\left({#1}, {#2}\right)}

%\newcommand{\myvec}[1]{\mathbf{#1}}
%\newcommand{\myvecsym}[1]{\boldsymbol{#1}}
\newcommand{\myvec}[1]{\mathbf{#1}}
\newcommand{\myvecsym}[1]{\boldsymbol{#1}}
\newcommand{\ind}[1]{\mathbb{I}(#1)}
%\newcommand{\ind}[1]{[#1]}


\newcommand{\vzero}{\myvecsym{0}}
\newcommand{\vone}{\myvecsym{1}}

\newcommand{\valpha}{\myvecsym{\alpha}}
\newcommand{\vbeta}{\myvecsym{\beta}}
\newcommand{\vchi}{\myvecsym{\chi}}
\newcommand{\vdelta}{\myvecsym{\delta}}
\newcommand{\vDelta}{\myvecsym{\Delta}}
\newcommand{\vepsilon}{\myvecsym{\epsilon}}
\newcommand{\vell}{\myvecsym{\ell}}
\newcommand{\veta}{\myvecsym{\eta}}
%\newcommand{\vEta}{\myvecsym{\Eta}}
\newcommand{\vgamma}{\myvecsym{\gamma}}
\newcommand{\vGamma}{\myvecsym{\Gamma}}
\newcommand{\vmu}{\myvecsym{\mu}}
\newcommand{\vmut}{\myvecsym{\tilde{\mu}}}
\newcommand{\vnu}{\myvecsym{\nu}}
\newcommand{\vkappa}{\myvecsym{\kappa}}
\newcommand{\vlambda}{\myvecsym{\lambda}}
\newcommand{\vLambda}{\myvecsym{\Lambda}}
\newcommand{\vLambdaBar}{\overline{\vLambda}}
\newcommand{\vomega}{\myvecsym{\omega}}
\newcommand{\vOmega}{\myvecsym{\Omega}}
\newcommand{\vphi}{\myvecsym{\phi}}
\newcommand{\vPhi}{\myvecsym{\Phi}}
\newcommand{\vpi}{\myvecsym{\pi}}
\newcommand{\vPi}{\myvecsym{\Pi}}
\newcommand{\vpsi}{\myvecsym{\psi}}
\newcommand{\vPsi}{\myvecsym{\Psi}}
\newcommand{\vtheta}{\myvecsym{\theta}}
\newcommand{\vthetat}{\myvecsym{\tilde{\theta}}}
\newcommand{\vTheta}{\myvecsym{\Theta}}
\newcommand{\vsigma}{\myvecsym{\sigma}}
\newcommand{\vSigma}{\myvecsym{\Sigma}}
\newcommand{\vSigmat}{\myvecsym{\tilde{\Sigma}}}
\newcommand{\vtau}{\myvecsym{\tau}}
\newcommand{\vxi}{\myvecsym{\xi}}

\newcommand{\vmuY}{\vb}
\newcommand{\vmuMu}{\vmu_{x}}
\newcommand{\vmuMuGivenY}{\vmu_{x|y}}
\newcommand{\vSigmaMu}{\vSigma_{x}}
\newcommand{\vSigmaMuInv}{\vSigma_{x}^{-1}}
\newcommand{\vSigmaMuGivenY}{\vSigma_{x|y}}
\newcommand{\vSigmaMuGivenYinv}{\vSigma_{x|y}^{-1}}
\newcommand{\vSigmaY}{\vSigma_{y}}
\newcommand{\vSigmaYinv}{\vSigma_{y}^{-1}}

%\newcommand{\vmuY}{\vmu_{y}}
%\newcommand{\vmuMu}{\vmu_{\mu}}
%\newcommand{\vmuMuGivenY}{\vmu_{\mu|y}}
%\newcommand{\vSigmaMu}{\vSigma_{\mu}}
%\newcommand{\vSigmaMuInv}{\vSigma_{\mu}^{-1}}
%\newcommand{\vSigmaMuGivenY}{\vSigma_{\mu|y}}
%\newcommand{\vSigmaMuGivenYinv}{\vSigma_{\mu|y}^{-1}}
%\newcommand{\vSigmaY}{\vSigma_{y}}
%\newcommand{\vSigmaYinv}{\vSigma_{y}^{-1}}

\newcommand{\muY}{\mu_{y}}
\newcommand{\muMu}{\mu_{\mu}}
\newcommand{\muMuGivenY}{\mu_{\mu|y}}
\newcommand{\SigmaMu}{\Sigma_{\mu}}
\newcommand{\SigmaMuInv}{\Sigma_{\mu}^{-1}}
\newcommand{\SigmaMuGivenY}{\Sigma_{\mu|y}}
\newcommand{\SigmaMuGivenYinv}{\Sigma_{\mu|y}^{-1}}
\newcommand{\SigmaY}{\Sigma_{y}}
\newcommand{\SigmaYinv}{\Sigma_{y}^{-1}}

\newcommand{\hatf}{\hat{f}}
\newcommand{\haty}{\hat{y}}
\newcommand{\const}{\mathrm{const}}
\newcommand{\sigmoid}{\mathrm{sigm}}

\newcommand{\one}{(1)}
\newcommand{\two}{(2)}

\newcommand{\va}{\myvec{a}}
\newcommand{\vb}{\myvec{b}}
\newcommand{\vc}{\myvec{c}}
\newcommand{\vd}{\myvec{d}}
\newcommand{\ve}{\myvec{e}}
\newcommand{\vf}{\myvec{f}}
\newcommand{\vg}{\myvec{g}}
\newcommand{\vh}{\myvec{h}}
\newcommand{\vj}{\myvec{j}}
\newcommand{\vk}{\myvec{k}}
\newcommand{\vl}{\myvec{l}}
\newcommand{\vm}{\myvec{m}}
\newcommand{\vn}{\myvec{n}}
\newcommand{\vo}{\myvec{o}}
\newcommand{\vp}{\myvec{p}}
\newcommand{\vq}{\myvec{q}}
\newcommand{\vr}{\myvec{r}}
\newcommand{\vs}{\myvec{s}}
\newcommand{\vt}{\myvec{t}}
\newcommand{\vu}{\myvec{u}}
\newcommand{\vv}{\myvec{v}}
\newcommand{\vw}{\myvec{w}}
\newcommand{\vws}{\vw_s}
\newcommand{\vwt}{\myvec{\tilde{w}}}
\newcommand{\vWt}{\myvec{\tilde{W}}}
\newcommand{\vwh}{\hat{\vw}}
\newcommand{\vx}{\myvec{x}}
%\newcommand{\vx}{\myvec{x}}
\newcommand{\vxt}{\myvec{\tilde{x}}}
\newcommand{\vy}{\myvec{y}}
\newcommand{\vyt}{\myvec{\tilde{y}}}
\newcommand{\vz}{\myvec{z}}

\newcommand{\vra}{\myvec{r}_a}
\newcommand{\vwa}{\myvec{w}_a}
\newcommand{\vXa}{\myvec{X}_a}


\newcommand{\vA}{\myvec{A}}
\newcommand{\vB}{\myvec{B}}
\newcommand{\vC}{\myvec{C}}
\newcommand{\vD}{\myvec{D}}
\newcommand{\vE}{\myvec{E}}
\newcommand{\vF}{\myvec{F}}
\newcommand{\vG}{\myvec{G}}
\newcommand{\vH}{\myvec{H}}
\newcommand{\vI}{\myvec{I}}
\newcommand{\vJ}{\myvec{J}}
\newcommand{\vK}{\myvec{K}}
\newcommand{\vL}{\myvec{L}}
\newcommand{\vM}{\myvec{M}}
\newcommand{\vMt}{\myvec{\tilde{M}}}
\newcommand{\vN}{\myvec{N}}
\newcommand{\vO}{\myvec{O}}
\newcommand{\vP}{\myvec{P}}
\newcommand{\vQ}{\myvec{Q}}
\newcommand{\vR}{\myvec{R}}
\newcommand{\vS}{\myvec{S}}
\newcommand{\vT}{\myvec{T}}
\newcommand{\vU}{\myvec{U}}
\newcommand{\vV}{\myvec{V}}
\newcommand{\vW}{\myvec{W}}
\newcommand{\vX}{\myvec{X}}
%\newcommand{\vXs}{\vX_{\vs}}
\newcommand{\vXs}{\vX_{s}}
\newcommand{\vXt}{\myvec{\tilde{X}}}
\newcommand{\vY}{\myvec{Y}}
\newcommand{\vZ}{\myvec{Z}}
\newcommand{\vZt}{\myvec{\tilde{Z}}}
\newcommand{\vzt}{\myvec{\tilde{z}}}

\newcommand{\vxtest}{\myvec{x}_*}
\newcommand{\vytest}{\myvec{y}_*}


\newcommand{\ftrue}{f_{true}}

\newcommand{\myprec}{\mathrm{prec}}
\newcommand{\precw}{\lambda_{w}} % precision of weights (alpha)
\newcommand{\precy}{\lambda_{y}} % precision of y (beta)
\newcommand{\fbar}{\overline{f}}
\newcommand{\xmybar}{\overline{x}}
\newcommand{\ybar}{\overline{y}}
\newcommand{\rbar}{\overline{r}}
\newcommand{\zbar}{\overline{z}}
\newcommand{\vAbar}{\overline{\vA}}
\newcommand{\vxbar}{\overline{\vx}}
\newcommand{\vXbar}{\overline{\vX}}
\newcommand{\vybar}{\overline{\vy}}
\newcommand{\vYbar}{\overline{\vY}}
\newcommand{\vzbar}{\overline{\vz}}
\newcommand{\vZbar}{\overline{\vZ}}
\newcommand{\xbar}{\overline{x}}
\newcommand{\wbar}{\overline{w}}
\newcommand{\Xbar}{\overline{X}}
\newcommand{\Ybar}{\overline{Y}}
\newcommand{\Gbar}{\overline{G}}
\newcommand{\Jbar}{\overline{J}}
\newcommand{\Lbar}{\overline{L}}
\newcommand{\Nbar}{\overline{N}}
%\newcommand{\Qbar}{\overline{Q}}
\newcommand{\Qbar}{\overline{Q}}
\newcommand{\Tbar}{\overline{T}}
\newcommand{\Sbar}{\overline{S}}
\newcommand{\vSbar}{\overline{\vS}}
\newcommand{\Rbar}{\overline{R}}

\newcommand{\vtaubar}{\overline{\vtau}}
\newcommand{\vtbar}{\overline{\vt}}
\newcommand{\vsbar}{\overline{\vs}}
\newcommand{\mubar}{\overline{\mu}}
\newcommand{\phibar}{\overline{\phi}}


\newcommand{\htilde}{\tilde{h}}
\newcommand{\vhtilde}{\tilde{\vh}}
\newcommand{\Dtilde}{\tilde{D}}
\newcommand{\Ftilde}{\tilde{F}}
\newcommand{\wtilde}{\tilde{w}}
\newcommand{\ptilde}{\tilde{p}}
\newcommand{\pstar}{p^*}
\newcommand{\xtilde}{\tilde{x}}
\newcommand{\Xtilde}{\tilde{X}}
\newcommand{\ytilde}{\tilde{y}}
\newcommand{\Ytilde}{\tilde{Y}}
\newcommand{\vxtilde}{\tilde{\vx}}
\newcommand{\vytilde}{\tilde{\vy}}
\newcommand{\ztilde}{\tilde{\z}}
\newcommand{\vthetaMAP}{\hat{\vtheta}_{MAP}}
\newcommand{\vthetaS}{\vtheta^{(s)}}
\newcommand{\vthetahat}{\hat{\vtheta}}
\newcommand{\thetahat}{\hat{\theta}}
\newcommand{\thetabar}{\overline{\theta}}
\newcommand{\vthetabar}{\overline{\vtheta}}
\newcommand{\pibar}{\overline{\pi}}
\newcommand{\vpibar}{\overline{\vpi}}



%\newcommand{\sss}{s^2}
%\newcommand{\vvv}{v}
\newcommand{\RSS}{\mathrm{RSS}}
\newcommand{\mydof}{\mathrm{dof}}



\newcommand{\vvec}{\mathrm{vec}}
\newcommand{\kron}{\otimes}
\newcommand{\dof}{\mathrm{dof}}
%\newcommand{\E}{E}
\newcommand{\E}{\mathbb{E}}
\newcommand{\energy}{E}
\newcommand{\expectAngle}[1]{\langle #1 \rangle}
\newcommand{\expect}[1]{\mathbb{E}\left[ {#1} \right]}
\newcommand{\expectQ}[2]{\mathbb{E}_{{#2}} \left[ {#1} \right]}
\newcommand{\Var}{\mathrm{Var}}
%\newcommand{\Var}{\mathbb{V}}
\newcommand{\var}[1]{\mathrm{var}\left[{#1}\right]}
\newcommand{\std}[1]{\mathrm{std}\left[{#1}\right]}
\newcommand{\varQ}[2]{\mathrm{var}_{{#2}}\left[{#1}\right]}
\newcommand{\cov}[1]{\mathrm{cov}\left[{#1}\right]}
\newcommand{\corr}[1]{\mathrm{corr}\left[{#1}\right]}
% \newcommand{\mode}[1]{\mathrm{mode}\left[{#1}\right]}
\newcommand{\median}[1]{\mathrm{median}\left[{#1}\right]}




\newcommand{\sech}{\mathrm{sech}}
%\newcommand{\cosh}{\mathrm{cosh}}
\newcommand{\kurt}{\mathrm{kurt}}
\newcommand{\proj}{\mathrm{proj}}
\newcommand{\myskew}{\mathrm{skew}}
\newcommand{\rank}{\mathrm{rank}}
\newcommand{\diag}{\mathrm{diag}}
\newcommand{\blkdiag}{\mathrm{blkdiag}}
\newcommand{\bias}{\mathrm{bias}}
%\newcommand{\dim}{\mathrm{dim}}
\newcommand{\union}{\cup}
\newcommand{\intersect}{\cap}



%\newcommand{\NN}{N}
%\newcommand{\NC}{N_C}
%\newcommand{\ND}{N_D}
%\newcommand{\NX}{N_X}
%\newcommand{\NXi}{N_{X_i}}
%\newcommand{\NY}{N_Y}
%\newcommand{\nx}{n_x}
%\newcommand{\ny}{n_y}
%\newcommand{\nv}{n_v}
%\newcommand{\nk}{n_k}


\newcommand{\myc}{c}
\newcommand{\myi}{i}
\newcommand{\myj}{j}
\newcommand{\myk}{k}
\newcommand{\myn}{n}
\newcommand{\myq}{q}
\newcommand{\mys}{s}
\newcommand{\myt}{t}



\newcommand{\kernelfn}{\kappa}

\newcommand{\Nsamples}{S}
\newcommand{\Ndata}{N}
\newcommand{\Ntrain}{N_{\mathrm{train}}}
\newcommand{\Ntest}{N_{\mathrm{test}}}
\newcommand{\Ndim}{D}
\newcommand{\Ndimx}{D_x}
\newcommand{\Ndimy}{D_y}
\newcommand{\Nhidden}{H}
\newcommand{\Noutdim}{D_y}
\newcommand{\Nlowdim}{L}
\newcommand{\Ndimlow}{L}
\newcommand{\Nstates}{K}
\newcommand{\Nfolds}{K}
\newcommand{\Npastates}{L}
\newcommand{\Nclasses}{C}
\newcommand{\Nclusters}{K}
\newcommand{\NclustersC}{C}
\newcommand{\Ntime}{T}
\newcommand{\Ntimes}{T}
\newcommand{\Niter}{T}
\newcommand{\Nnodes}{D}

\newcommand{\assign}{\leftarrow}



%\newcommand{\xdi}{x_{di}}
%\newcommand{\xji}{x_{ji}}
%\newcommand{\yi}{y_i}



\newcommand{\ki}{i}
\newcommand{\kj}{j}
\newcommand{\kk}{k}
\newcommand{\kC}{C}
\newcommand{\kc}{c}

\newcommand{\supp}{\mathrm{supp}}
\newcommand{\query}{\calQ}
\newcommand{\vis}{\calE}
\newcommand{\nuisance}{\calN}
\newcommand{\hid}{\calH}

\newcommand{\advanced}{*}
%\newcommand{\advanced}{}



\newcommand{\bbI}{\mathbb{I}}
\newcommand{\bbL}{\mathbb{L}}
\newcommand{\bbM}{\mathbb{M}}
\newcommand{\bbS}{\mathbb{S}}


\newcommand{\calA}{{\cal A}}
\newcommand{\calB}{{\cal B}}
\newcommand{\calC}{{\cal C}}
\newcommand{\calD}{{\cal D}}
\newcommand{\calDx}{{\cal D}_x}
\newcommand{\calE}{{\cal E}}
\newcommand{\cale}{{\cal e}}
\newcommand{\calF}{{\cal F}}
\newcommand{\calG}{{\cal G}}
\newcommand{\calH}{{\cal H}}
\newcommand{\calHX}{{\cal H}_X}
\newcommand{\calHy}{{\cal H}_y}
\newcommand{\calI}{{\cal I}}
\newcommand{\calK}{{\cal K}}
\newcommand{\calM}{{\cal M}}
\newcommand{\calN}{{\cal N}}
\newcommand{\caln}{{\cal n}}
\newcommand{\calNP}{{\cal NP}}
\newcommand{\calMp}{\calM^+}
\newcommand{\calMm}{\calM^-}
\newcommand{\calMo}{\calM^o}
\newcommand{\Ctest}{C_*}
\newcommand{\calL}{{\cal L}}
\newcommand{\calP}{{\cal P}}
\newcommand{\calq}{{\cal q}}
\newcommand{\calQ}{{\cal Q}}
\newcommand{\calR}{{\cal R}}
\newcommand{\calS}{{\cal S}}
\newcommand{\calSstar}{\calS_*}
\newcommand{\calT}{{\cal T}}
\newcommand{\calV}{{\cal V}}
\newcommand{\calv}{{\cal v}}
\newcommand{\calX}{{\cal X}}
\newcommand{\calY}{{\cal Y}}

\newcommand{\Lone}{$\ell_1$}
\newcommand{\Ltwo}{$\ell_2$}

%\newcommand{\mya}{\mbox{a}}
%\newcommand{\myat}{\alpha_{t|t-1}}
\newcommand{\score}{\mbox{score}}
\newcommand{\AIC}{\mbox{AIC}}
\newcommand{\BIC}{\mbox{BIC}}
\newcommand{\BICcost}{\mbox{BIC-cost}}
\newcommand{\scoreBIC}{\mbox{score-BIC}}
\newcommand{\scoreBICL}{\mbox{score-BIC-L1}}
\newcommand{\scoreL}{\mbox{score-L1}}

\newcommand{\ecoli}{\mbox{{\it E. coli}}}
\newcommand{\doPearl}{\mathrm{do}}
\newcommand{\data}{\calD}
\newcommand{\model}{\calM}
\newcommand{\dataTrain}{\calD_{\mathrm{train}}}
\newcommand{\dataTest}{\calD_{\mathrm{test}}}
\newcommand{\dataValid}{\calD_{\mathrm{valid}}}
\newcommand{\Xtrain}{\vX_{\mathrm{train}}}
\newcommand{\Xtest}{\vX_{\mathrm{test}}}
\newcommand{\futuredata}{\tilde{\calD}}
\newcommand{\algo}{\calA}
\newcommand{\fitAlgo}{\calF}
\newcommand{\predictAlgo}{\calP}
%\newcommand{\data}{D}
\newcommand{\err}{\mathrm{err}}
\newcommand{\logit}{\mathrm{logit}}

% graph terms 
\newcommand{\nbd}{\mathrm{nbd}}
\newcommand{\nbr}{\mathrm{nbr}}
\newcommand{\anc}{\mathrm{anc}}
\newcommand{\desc}{\mathrm{desc}}
\newcommand{\pred}{\mathrm{pred}}
\newcommand{\mysucc}{\mathrm{suc}}
\newcommand{\nondesc}{\mathrm{nd}}
\newcommand{\pa}{\mathrm{pa}}
%\newcommand{\pa}{\pi}
\newcommand{\parent}{\mathrm{pa}}
\newcommand{\copa}{\mathrm{copa}}
\newcommand{\ch}{\mathrm{ch}}
\newcommand{\mb}{\mathrm{mb}}
\newcommand{\connects}{\sim}
\newcommand{\nd}{\mathrm{nd}}
\newcommand{\bd}{\mathrm{bd}}
\newcommand{\cl}{\mathrm{cl}}



\newcommand{\be}{\begin{equation}}
\newcommand{\ee}{\end{equation}}
\newcommand{\bea}{\begin{eqnarray}}
\newcommand{\eea}{\end{eqnarray}}
\newcommand{\beaa}{\begin{eqnarray*}}
\newcommand{\eeaa}{\end{eqnarray*}}

%%%%%%%%%%% Hoyt

\newcommand{\conv}[1]{\,\,\,\displaystyle{\operatorname*{\longrightarrow}^{\,_{#1}\,}}\,\,\,}
\newcommand{\dconv}{\conv{D}}
\newcommand{\pconv}{\conv{P}}
\newcommand{\asconv}{\conv{AS}}
\newcommand{\lpconv}[1]{\conv{L^{#1}}}

\DeclareMathAlphabet{\mathpzc}{OT1}{pzc}{m}{n}
%\newcommand{\inv}[1]{\ensuremath{\frac{1}{#1}}}
%\newcommand{\T}[1]{{\ensuremath{\left(#1\right)}}}
%\newcommand{\Tbr}[1]{{\ensuremath{\left[#1\right]}}}
%\newcommand{\Normal}[1]{\ensuremath{\mathpzc{N}\T{#1}}}
%\newcommand{\expof}[1]{\ensuremath{\exp\Tbr{#1}}}
%\newcommand{\So}{\ensuremath{\Rightarrow}}
%\newcommand{\ud}{\ensuremath{\mathrm{d}}}



\newcommand{\vfj}{\vf_j}
\newcommand{\vfk}{\vf_k}

\newcommand{\entropyBethe}{\mathbb{H}_{\mathrm{Bethe}}}
\newcommand{\entropyKikuchi}{\mathbb{H}_{\mathrm{Kikuchi}}}
\newcommand{\entropyEP}{\mathbb{H}_{\mathrm{ep}}}
\newcommand{\entropyConvex}{\mathbb{H}_{\mathrm{Convex}}}

\newcommand{\freeEnergyBethe}{F_{\mathrm{Bethe}}}
\newcommand{\freeEnergyKikuchi}{F_{\mathrm{Kikuchi}}}
\newcommand{\freeEnergyConvex}{F_{\mathrm{Convex}}}

\newcommand{\sigmaMle}{\hat{\sigma}^2_{mle}}
\newcommand{\sigmaUnb}{\hat{\sigma}^2_{unb}}


%**********************************
\newcommand{\keywordSpecial}[2]{{\bf #1}\index{keywords}{#2@#1}}
\newcommand{\bfidx}[1]{{\bf #1}}
%\newcommand{\keywordDef}[1]{{\bf #1}\index{keywords}{#1|bfidx}}
\newcommand{\keywordDefSpecial}[2]{{\bf #1}\index{keywords}{#2@#1|bfidx}}

\newcommand{\keywordDef}[1]{{\color{Blue}{\bf #1}}}


%\newcommand{\keywordDef}[1]{{\bf #1}}
\usepackage{tikz}
\usepackage{algorithm}
\usepackage{algorithmic}
\usepackage{graphicx} 
\usepackage{etoolbox}
\usepackage{xcolor}
% Theme for beamer presentation.
\setbeamertemplate{footline}[page number]{}
\setbeamertemplate{headline}{}
\setbeamertemplate{navigation symbols}{}

\definecolor{myColor}{rgb}{0.1,0.0,0.8}

\setbeamertemplate{theorems}[numbered]
\pretocmd{\part}{\setcounter{theorem}{0}}{}{}
\newtheorem{proposition}[theorem]{Proposition}

\DeclareMathOperator*{\argmax}{arg\,max}

% \usepackage{Antibes} 
\usepackage{seahorse}
% Other themes include: beamerthemetree, beamerthemelined, 
%                       beamerthemetree, beamerthemetreebars  

\title{Bayesian Optimization in High Dimensions via Random Embeddings}    
\author[Ziyu Wang]{Ziyu Wang\\ [3mm]Joint work with Masrour Zoghi, Frank Hutter, 
David Matheson, Nando de Freitas}

\date{}                    % Enter the date or \today between curly braces



% NOTE:  Joint work with Prof. Nando de Freitas

\begin{document}

% Creates title page of slide show using above information
\begin{frame}
  \titlepage
\end{frame}
\note{Talk for 20 minutes} % Add notes to yourself that will be displayed when
                           % typeset with the notes or notesonly class options

\section[Outline]{}

% Creates table of contents slide incorporating
% all \section and \subsection commands
\begin{frame}
  \tableofcontents
\end{frame}


\section{Motivation}
\label{sec:ahmc}
\begin{frame}<beamer>
 \tableofcontents[currentsection]
\end{frame}

\begin{frame}
 \frametitle{Bayesian Opitmization}
 {\bf \textcolor{myColor}{Bayesian Optimization (BO)}}

 Let $f: {\cal X} \to \mathbb{R}$ be a function on a compact subset 
 ${\cal X} \subseteq \mathbb{R}^D$. 
 BO addresses the following global optimization problem
 \[ \vx^{\star} = \argmax_{\vx \in {\cal X}} f(\vx). \]

 We are particularly interested in objective functions $f$ 
 that may satisfy one or more of the following criteria: 
 \begin{itemize}
  \item noisy,
  \item expensive to evaluate,
  \item do not have easily available derivatives.
 \end{itemize}
\end{frame}

\begin{frame}
 \frametitle{Bayesian Opitmization}
 \begin{itemize}
  \item BO uses a prior distribution 
  that captures our beliefs about the behavior of $f$.
  \item It then updates this prior with sequentially acquired data.
 \end{itemize}

 Specifically, it iterates the following phases:
 \begin{enumerate}
  \item use the prior to decide at which input $x\in \cal X$ to query $f$ next
  by optimizing {\bf \textcolor{myColor}{aquisition functions}};
  \item evaluate $f(x)$;
  \item update the prior based on the new data $\langle{}x, f(x)\rangle$.
 \end{enumerate}
\end{frame}

 \begin{frame}
 \frametitle{Bayesian Opitmization}
  \begin{figure}
   \centering
   \includegraphics[width=0.75\columnwidth]
   {./figs/bo}
   \label{fig:traj}
  \end{figure}
 \end{frame}

 \begin{frame}
 \frametitle{Curse of Dimensionality}
  \begin{itemize}
   \item In recent years, the artificial intelligence community has increasingly used Bayesian optimization; see for example~\cite{Martinez-Cantin:2009,Brochu:2009,Srinivas:2010,Hoffman:2011,Lizotte:2011,Azimi:2012}
   \item But BO is an {\bf \textcolor{myColor}{global}} 
    optimization algorithm. Thus it has to 
    {\bf \textcolor{myColor}{explore throughly}}.
   \item But the volume to be explored increases 
   {\bf \textcolor{myColor}{exponentially}} with dimensionality.
   \item In high dimensional problems, BO may explore too much!
   \item Thus we are cursed!
   
  \end{itemize}
 \end{frame}

 

\section{REMBO}
 \begin{frame}<beamer>
  \tableofcontents[currentsection]
 \end{frame}
   
 \begin{frame}
   \frametitle{Irrelevant dimensions}
    \begin{minipage}[l]{0.5\columnwidth}
     \begin{itemize}
      \item Many researchers have noted that for certain classes of problems 
       most dimensions do not change the objective function significantly
       ~\cite{Bergstra:2012,Hutter:2013_KeyParameters}.
      \item That is to say these problems have 
       {\bf \textcolor{myColor}{low effective dimensionality}}
     \end{itemize}

    \end{minipage}
    \begin{minipage}[r]{0.485\columnwidth}
     \begin{figure}[t]
      \includegraphics[width = 1.2\columnwidth]
      {./figs/irrelevant}
      \label{fig:ESSL_BLR}
     \end{figure}
    \end{minipage}
  \end{frame}

 \begin{frame}
  \frametitle{Random Embedding}
  To take advantage of low effective dimensionality, we propose to 
  randomly {\bf \textcolor{myColor}{embed}} 
  a low dimensional space into the high dimensional space and optimize
  only on the low dimensional space.
  \begin{figure}[t]
   \includegraphics[width = 0.9\columnwidth]
   {../paper/figures/2to1embedding}
   \label{fig:ESSL_BLR}
  \end{figure}
 \end{frame}
 
 \begin{frame}
  \frametitle{REMBO}
  \begin{itemize}
   \item Choose compact set {\bf \textcolor{myColor}{$\mathcal{Y}$}}.
   \item Draw a random Gaussian matrix A.
   \item Repeat:
   \begin{enumerate}
    \item Use the prior to decide at which input $y \in \cal Y$ to query $f$ next
    by optimizing aquisition functions;
    \item Evaluate {\bf \textcolor{myColor}{$f(Ay)$}};
    \item Update the prior based on the new data 
    {\bf \textcolor{myColor}{$\langle{}y, f(Ay)\rangle$}}.
   \end{enumerate}
  \end{itemize}
 \end{frame}

 \begin{frame}
  \begin{figure}[t!]
\centering
  \includegraphics[scale=0.28]{../paper/figures/projection.pdf}
  \caption{Embedding from $d=1$ into $D=2$. The box illustrates the 2D constrained space ${\cal X}$, while the thicker red line illustrates the 1D constrained space $\mathcal{Y}$. {\bf \textcolor{myColor}{The set $\mathcal{Y}$ must be chosen large enough}} so that the projection of its image, $\vA \mathcal{Y}$, onto the effective subspace (vertical axis in this diagram) covers the vertical side of the box.}
  \label{fig:proj}
  \vspace{-1em}
\end{figure}
 \end{frame}
 
 \begin{frame}
  \frametitle{Guarantees}
  \begin{theorem}
   \label{prop:1}
   Assume we are given a function $f: \mathbb{R}^{D} \rightarrow \mathbb{R}$ with effective dimensionality $d_e$ and a random matrix $\vA \in \mathbb{R}^{D\times d}$ with independent entries sampled according to $\mathcal{N}(0, 1)$ and $d\geq d_e$. Then, with probability 1, for any $\vx \in \mathbb{R}^D$, there exists a $ \vy \in \mathbb{R}^d$ such that $f(\vx) = f(\vA\vy)$.
  \end{theorem}
 \end{frame}
 
 \begin{frame}
  \frametitle{Setting $\cal Y$}
  \begin{theorem}
   \label{prop:2}
   Suppose we want to optimize a function $f: \mathbb{R}^{D} \rightarrow \mathbb{R}$ with effective dimension $d_e \leq d$ subject to the box constraint $\mathcal{X} \subset \mathbb{R}^D$, where $\mathcal{X}$ is centered around $\mathbf{0}$. Let us denote one of the optimizers by $\vx^{\star}$.
   Suppose further that the effective subspace $\cal T$ of $f$ is such that $\cal T$ is the span of $d_e$ basis vectors. 
   Let $\vx^{\star}_\top \in \cal{T} \cap \mathcal{X}$ be an optimizer of $f$ inside $\mathcal{T}$. 
   If $\vA$ is a $D\times d$ random matrix with independent standard Gaussian entries,
   there exists an optimizer $\vy^\star \in \mathbb{R}^{d}$ such that $f(\vA\vy^\star) = f(\vx^\star_\top)$ and $\|\vy^\star\|_2 \leq \frac{\sqrt{d_e}}{\epsilon}\|\vx^{\star}_\top\|_2$ with probability at least $1-\epsilon$.
 \end{theorem}
 \end{frame}
 
 \begin{frame}
  \frametitle{Low VS. High dimensional Kernel}
  \begin{itemize}
   \item {\bf \textcolor{myColor}{low-dimensional kernel}} (defined on 
   {\bf \textcolor{myColor}{$\cal Y$}}):
  \[ k_{\ell}^d(\vy^{(1)},\vy^{(2)}) = \exp\left({-\frac{\|\vy^{(1)}-\vy^{(2)}\|^2}{2\ell^2}}\right). \]
   \item {\bf \textcolor{myColor}{high-dimensional kernel}} (defined on 
   {\bf \textcolor{myColor}{$\cal X$}}):  
  $$k_{\ell}^D(\vy^{(1)}, \vy^2) = \exp\left( -\frac{\| p_{\mathcal{X}}(\vA\vy^{(1)}) - p_{\mathcal{X}}(\vA\vy^{(2)}) \|^2}{2\ell^2} \right),$$
where $p_{\mathcal{X}}:\mathbb{R}^D \rightarrow \mathbb{R}^D$ is the standard projection operator for our box-constraint: $p_{\mathcal{X}}(\vy) = {\arg \min}_{\vz\in \mathcal{X}} \|\vz-\vy\|_2$; . 
  \end{itemize}

 \end{frame}


 \section{Experiments}
 \begin{frame}<beamer>
  \tableofcontents[currentsection]
 \end{frame}
 
 \begin{frame}<beamer>
  \frametitle{BO in 1,000,000,000 dimensions}
  \begin{figure}
%    \includegraphics[width=0.31\columnwidth]{../paper/figures/branin_dis_25.png}
%    \includegraphics[width=0.31\columnwidth]{../paper/figures/branin_dis_rot.png}
   \includegraphics[width=0.6\columnwidth]{../paper/figures/branin_dis_1b.png}
   \caption{Comparison of random search (RANDOM) and REMBO
     in $D=10^9$ extrinsic dimensions. 
     We plot means and $1/4$ standard deviation confidence intervals of the optimality gap across $50$ trials.}
   \label{fig:standard}
  \end{figure}
 \end{frame}
 
 \begin{frame}<beamer>
  \frametitle{Invariant to rotation}
  
  
  \begin{figure}
%    \includegraphics[width=0.31\columnwidth]{../paper/figures/branin_dis_25.png}
   \includegraphics[width=0.6\columnwidth]{../paper/figures/branin_dis_rot.png}
%    \includegraphics[width=0.6\columnwidth]{../paper/figures/branin_dis_1b.png}
   \caption{Comparison of random search (RANDOM), Bayesian optimization (BO),
     method by~\protect\cite{Chen:2012} (HD BO), and REMBO.
     $D=25$, with a rotated objective function. We plot means and $1/4$ standard deviation confidence intervals of the optimality gap across 50 trials.}
   \label{fig:standard}
  \end{figure}
 \end{frame}

 \begin{frame}<beamer>
 \frametitle{Automatic Configuration of a Mixed Integer
Linear Programming Solver}
  \begin{figure}[h!]
   \begin{center}
     \includegraphics[scale=0.35]{../paper/figures/lpsolve.png}\\
%      \includegraphics[scale=0.35]{../paper/figures/lpsolve_interleave.png}
     \caption{Performance for configuration of \texttt{lpsolve}; we show the optimality gap \texttt{lpsolve} achieved with the configurations found by the various methods (lower is better). 
%      Top: a single run of each method; Bottom: performance with $k=4$ interleaved runs. We plot means and $1/4$ standard deviations over 20 repetitions of the experiment.
           %results of 20  confidence. Note that the standard deviation of the mean would be slightly smaller than the ones plotted as we repeat the experiment 20 times.
           }
   \label{fig:lpsolve}
   \end{center}
   \vspace*{-3mm}
  \end{figure}
 \end{frame}

 \begin{frame}{Bibliography}
  \tiny
  \bibliographystyle{plain}
  \bibliography{../paper/bayesopt}
 \end{frame}
 
 \begin{frame}<beamer>
 \frametitle{Thank You!}
  \begin{center}
  \huge
    Questions?
  \end{center}
 \end{frame}
 
 

 

\end{document}